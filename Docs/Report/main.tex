\documentclass[12pt]{article}
\usepackage[pdftex]{graphicx} % Required for inserting images
\usepackage[T1]{fontenc}
\usepackage[polish]{babel}
\usepackage{amsmath}
\usepackage{geometry}
\usepackage{float}
\usepackage{hyperref}
\usepackage{polski}
\usepackage[utf8]{inputenc}
\usepackage{gensymb}
\usepackage{geometry}
\usepackage{longtable}
\usepackage{array} % required for text wrapping in tables
\usepackage{algpseudocode}
\usepackage{algorithm}
\usepackage{indentfirst}
\usepackage{verbatim}
\usepackage{amsthm}
\usepackage{thmtools}
\theoremstyle{definition}
\newtheorem{definition}{Definicja}
\begin{document}

\title{%
  \textbf{Teoria algorytmów i obliczeń} \\
  \large Projekt zaliczeniowy}

\author
{
    Piotr Jacak \\
    Jakub Kindracki \\
    Wiktor Kobielski \\
    Ernest Mołczan \\
    \\
    Koordynator: prof. dr hab. inż. Władysław Homenda
    \\
    \\
}

\date{Semestr zimowy 2025/2026}



\begin{figure}
    \centering
    \includegraphics[width=1\linewidth]{mini.png}
\end{figure}

\maketitle

\pagebreak
\tableofcontents
\pagebreak

\section{Wstęp}
\section{Definicje pojęć}
\section{Rozmiar multigrafu}
\section{Odległość multigrafów}
\section{Minimalne rozszerzenie multigrafu}
\section{Testy}
\section{Podsumowanie}
\section{Bibliografia}

\bibliographystyle{plain}
\bibliography{refs}

\end{document}
\documentclass[12pt]{article}
\usepackage[pdftex]{graphicx} % Required for inserting images
\usepackage[T1]{fontenc}
\usepackage[polish]{babel}
\usepackage{amsmath}
\usepackage{geometry}
\usepackage{float}
\usepackage{hyperref}
\usepackage{polski}
\usepackage[utf8]{inputenc}
\usepackage{gensymb}
\usepackage{geometry}
\usepackage{longtable}
\usepackage{array} % required for text wrapping in tables
\usepackage{algpseudocode}
\usepackage{algorithm}
\usepackage{indentfirst}
\usepackage{verbatim}
\usepackage{amsthm}
\usepackage{amsfonts}
\usepackage{thmtools}
\theoremstyle{definition}
\newtheorem{definition}{Definicja}
\begin{document}

\title{
  \textbf{Teoria algorytmów i obliczeń} \\
  \large Projekt zaliczeniowy}

\author
{
    Piotr Jacak \\
    Jakub Kindracki \\
    Wiktor Kobielski \\
    Ernest Mołczan \\
    \\
    Koordynator: prof. dr hab. inż. Władysław Homenda
    \\
    \\
}

\date{Semestr zimowy 2025/2026}



\begin{figure}
    \centering
    \includegraphics[width=1\linewidth]{mini.png}
\end{figure}

\maketitle

\pagebreak
\tableofcontents
\pagebreak

% ---------------- WSTĘP ---------------- %
\section{Wstęp}
\label{sec:wstep} % Etykieta do późniejszych odwołań, np. \ref{sec:wstep}

Niniejsza praca stanowi sprawozdanie z projektu zrealizowanego w ramach przedmiotu \textbf{Teoria algorytmów i obliczeń}. Przedmiotem badań są algorytmy operujące na multigrafach, ze szczególnym uwzględnieniem problematyki izomorfizmu podgrafów oraz minimalnych rozszerzeń grafów.

Głównym celem projektu jest opracowanie, analiza teoretyczna oraz implementacja algorytmów rozwiązujących dwa ściśle powiązane problemy. Pierwszym z nich jest weryfikacja, czy dany multigraf $H$ jest izomorficzny z podgrafem multigrafu $G$. Drugim, kluczowym zagadnieniem, jest wyznaczenie \emph{minimalnego rozszerzenia} multigrafu $G$ do postaci $G'$, która zawiera co najmniej jeden podgraf izomorficzny z $H$.

Realizacja powyższych celów wymagała formalnego zdefiniowania oraz uzasadnienia kilku fundamentalnych pojęć. W pracy zaproponowano autorskie lub bazujące na literaturze definicje:
\begin{itemize}
    \item \emph{rozmiaru multigrafu},
    \item \emph{metryki} w zbiorze multigrafów,
    \item \emph{minimalnego rozszerzenia} multigrafu.
\end{itemize}
Pojęcia te stanowią podstawę do dalszej analizy algorytmicznej oraz oceny kosztu operacji.

W ramach pracy przeprowadzono analizę złożoności obliczeniowej opracowanych algorytmów. Zgodnie z założeniami projektu, w przypadku gdy algorytmy dokładne charakteryzują się złożonością wykładniczą, przedstawiono również propozycje algorytmów aproksymacyjnych o złożoności wielomianowej.

Niniejszy raport, oprócz formalnych definicji i analizy algorytmów, zawiera także opis przeprowadzonych testów obliczeniowych, dokumentację techniczną implementacji oraz wnioski końcowe.

\pagebreak
% ---------------- WSTĘP ---------------- %

% ---------------- DEFINICJE ---------------- %
\section{Definicje pojęć}
\label{sec:definicje}

% Definicje grafu, multigrafu, izomorfizmu, podgrafu izomorficznego, macierzy sąsiedztwa

\begin{definition}[Graf]
    Grafem nazywamy parę $G = (V, E)$, gdzie $V$ jest zbiorem wierzchołków, a $E \subseteq V \times V = \{(u, v) : u, v \in V \land u \neq v \}$ jest zbiorem krawędzi. Dla każdej pary wierzchołków $u, v \in V$ istnieje co najwyżej jedna krawędź łącząca wierzchołki $u$ i $v$.
\end{definition}

\begin{definition}[Multigraf]
    Multigrafem nazywamy graf, w którym pomiędzy dowolnymi dwoma różnymi wierzchołkami $u, v \in V$ może istnieć więcej niż jedna krawędź.
\end{definition}

\begin{definition}[Izomorfizm grafów]
    Dwa grafy $G_1 = (V_1, E_1)$ i $G_2 = (V_2, E_2)$ są izomorficzne, wtedy i tylko wtedy, gdy istnieje bijekcja $f: V_1 \to V_2$, taka że dla każdej krawędzi $(u, v) \in E_1$ zachodzi $(f(u), f(v)) \in E_2$. Definicja ta jest analogiczna dla multigrafów.
\end{definition}

\begin{definition}[Podgraf]
    Graf $H = (V_H, E_H)$ nazywamy podgrafem grafu $G = (V_G, E_G)$, wtedy i tylko wtedy, gdy $V_H \subseteq V_G$ oraz $E_H \subseteq E_G$. Definicja ta jest analogiczna dla multigrafów.
\end{definition}

\begin{definition}[Graf atrybutowy]
    Graf $G = (V, E, f)$ nazywamy grafem atrybutowym, gdzie $V$ jest zbiorem wierzchołków, a $E \subseteq V \times V = \{(u, v) : u, v \in V \land u \neq v \}$ jest zbiorem krawędzi. Dla każdej pary wierzchołków $u, v \in V$ istnieje co najwyżej jedna krawędź łącząca wierzchołki $u$ i $v$. $f: E \to \Sigma_E $ jest funkcją, przypisującą etykiety wszystkim krawędziom w grafie $G$. 
\end{definition}

\begin{definition}[Macierz sąsiedztwa]
    Macierzą sąsiedztwa multigrafu $G = (V, E)$ nazywamy macierz $A$, której pole $A_{uv} = k$, wtedy i tylko wtedy, gdy istnieje $k$ krawędzi $(u, v) \in E$. W przypadku gdy nie istnieje żadna krawędź pomiędzy wierzchołkami $u$ i $v$, to $A_{uv} = 0$. Dla zwykłych grafów, macierz sąsiedztwa jest macierzą binarną.
\end{definition}

\pagebreak
% ---------------- DEFINICJE ---------------- %

% ---------------- ROZMIAR MULTIGRAFU ---------------- %
\section{Rozmiar multigrafu}
\label{sec:rozmiar}

\begin{definition}[Rozmiar multigrafu]
    Rozmiarem $S$ multigrafu $G = (V, E)$ nazywamy sumę liczby wierzchołków $|V|$ oraz liczby krawędzi $|E|$ grafu $G$:
    \[ S(G) = |V| + |E| \]
\end{definition}

\begin{definition}[Porządek w zbiorze wszystkich multigrafów]
    Niech $G_1$ i $G_2$ będą dwoma multigrafami. Mówimy, że $G_1$ jest mniejszy, lub równy $G_2$ wtedy i tylko wtedy, gdy suma liczb wierzchołków i krawędzi grafu $G_1$ jest mniejsza, lub równa sumie liczb wierzchołków i krawędzi grafu $G_2$, czyli $S(G_1) \leq S(G_2)$.
\end{definition}

Żeby udowodnić poprawność powyższej definicji porządku wykazujemy, że spełnia ona trzy wymagane własności:
\begin{itemize}
    \item \textbf{Zwrotność}: \\
    Dla każdego multigrafu $G$, $S(G) = S(G)$. Jest to prawda, ponieważ suma liczby wierzchołków i krawędzi multigrafu $G$ jest równa samej sobie.
    \item \textbf{Przechodniość}: \\
    Dla dowolnych multigrafów $G_1$, $G_2$ oraz $G_3$, jeśli $S(G_1) \leq S(G_2)$ oraz $S(G_2) \leq S(G_3)$, to $S(G_1) \leq S(G_3)$. Jest to oczywiście prawda.
    \item \textbf{Antysymetryczność}: Dla dowolnych multigrafów $G_1$ oraz $G_2$, jeśli $S(G_1) \leq S(G_2)$ oraz $S(G_2) \leq S(G_1)$, to $G_1$ jest równy w sensie wcześniej zdefiniowanego rozmiaru z $G_2$, czyli $S(G_1) = S(G_2)$.
\end{itemize}

\pagebreak
% ---------------- ROZMIAR MULTIGRAFU ---------------- %

% ---------------- METRYKA W ZBIORZE WSZYSTKICH MULTIGRAFÓW ---------------- %
\section{Metryka w zbiorze wszystkich multigrafów}
\label{sec:metryka}

\pagebreak
% ---------------- METRYKA W ZBIORZE WSZYSTKICH MULTIGRAFÓW ---------------- %

% ---------------- MINIMALNE ROZSZERZENIE MULTIGRAFU ---------------- %
\section{Minimalne rozszerzenie multigrafu}
\label{sec:minimalne_rozszerzenie}

% ALGORYTM APROKSYMACYJNY DO PROBLEMU IZOMORFIZMU PODGRAFU
\subsection{Algorytm aproksymacyjny do problemu izomorfizmu podgrafu}
Do problemu można zastosować algorytm LeRP (Length-R Paths) opracowany przez Freda W DePiero oraz Davida Krouta \cite{lerp_algorithm}. Autorzy bezpośrednio stwierdzają, że nie zajmują się multigrafami. Zaznaczają natomiast, że multigrafy nie są ograniczeniem, ponieważ można stworzyć reprezentację multigrafu opisującą istnienie równoległych krawędzi za pomocą schematu kolorowania krawędzi. Zatem kompletny algorytm aproksymacyjny będzie składał się z dwóch kroków:
\begin{enumerate}
    \item Redukcja: Transfomacja multigrafów wejściowych $G_1$ oraz $G_2$ na grafy atrybutowe $G'_1$ oraz $G'_2$.
    \item Aproksymacja: Wykorzystanie algorytmu LeRP (który obsługuje grafy atrybutowe) do stwierdzenia, czy graf $G_1$ jest izomorficzny z pewnym podgrafem $P$ grafu $G_2$.
\end{enumerate}

\subsubsection{Transformacja multigrafu}
Niech $G_1$ i $G_2$ będą multigrafami. Odpowiadające im grafy atrybutowe $G'_1$ i $G'_2$ konstruowane są w następujący sposób:
\begin{itemize}
    \item Zbiory wierzchołków pozostają bez zmian ($V'_1 = V_1, V'_2 = V_2$).
    \item Dla każdej pary wierzchołków $(u, v)$ w $G_1$ (i analogicznie w $G_2$): Jeśli między $u$ a $v$ w $G_1$ istnieje $k$ równoległych krawędzi, to w $G'_1$ tworzona jest pojedyncza krawędź $(u, v)$ z atrybutem $k \in \mathbb{N^+}$.
\end{itemize}

\subsubsection{Aproksymacja algorytmem LeRP}
Fundamentalna zasada LeRP różni się od algorytmów backtrackingu (przykładowo algorytm Ullmanna). Zamiast próbować dopasować pełną strukturę sąsiedztwa krok po kroku, LeRP opiera się na założeniu, że o podobieństwie strukturalnym dwóch wierzchołków można wnioskować na podstawie porównania liczby ścieżek (\textit{sygnatur}) o różnej długości ($r$ - parametr algorytmu) w ich sąsiedztwie. Działanie algorytmu LeRP można podzielić na kilka etapów, realizowanych na grafach atrybutowych $G'_1$ i $G'_2$. 
\begin{enumerate}
    \item Obliczanie liczby ścieżek (pre-processing) \\
    Dla obu grafów $G'_1$ i $G'_2$ obliczane są potęgi ich macierzy sąsiedztwa, odpowiednio $A^r$ i $B^r$ aż do maksymalnej długości $R$. Wartość $A_{ij}^r$ w macierzy $A^r$ reprezentuje liczbę ścieżek o długości dokładnie $r$ z wierzchołka $i$ do wierzchołka $j$. W kontekście omówionej transformacji, macierz $A$ jest macierzą, gdzie $A_{ij}$ przechowuje atrybut $k$ - liczbę równoległych krawędzi wcześniejszego multigrafu $G_1$. Obliczenie $A^r$ polega na $r$-krotnym mnożeniu macierzy.
    \item Porównanie strukturalne \\
    Dla każdej pary wierzchołków $g_{1i} \in G'_1$ i $g_{2k} \in G'_2$ algorytm porównuje ich struktury: sprawdza, do jakiej długości ścieżki ($r = 1, ..., R$) grafy wyglądają tak samo. Im dłużej są podobne, tym wyższy wynik podobieństwa $r_{max}$.
    \item Mapping \\
    Algorytm zaczyna od pojedynczego ziarna mapowania - wstępnie przypisuje wierzchołek $g_{1i}$ do wierzchołka $g_{2k}$. Następnie iteracyjnie dodaje do tego mapowania sąsiadów już zamapowanych wierzchołków, wybierając te, które maksymalizują wskaźnik podobieństwa i są spójne ze znalezionymi wcześniej. 
\end{enumerate}
LeRP opiera się na założeniu, że dopasowanie sygnatur ścieżek (taka sama liczba ścieżek o długości $1, 2, ..., R$ jest wystarczającym dowodem podobieństwa, aby utworzyć mapowanie. \\ \\ 
Złożoność pesymistyczna algorytmu LeRP jest wielomianowa i wynosi $O(N^3 \cdot D^2 \cdot R)$, gdzie:
\begin{itemize}
    \item $N$ to liczba wierzchołków w grafach (zakładając, że oba grafy mają rozmiar rzędu $N$).
    \item $D$ to średni stopień wierzchołków w grafach.
    \item $R$ to maksymalna długość ścieżki brana pod uwagę. 
\end{itemize}
Uzasadnienie złożoności: porównanie par wierzchołków wymaga $N^2$ porównań. Każdy wierzchołek ma średnio $D$ sąsiadów, więc daje to $D^2$ dodatkowych operacji. Analiza różnych długości ścieżek to czynnik $R$. Podczas budowania dopasowania, algorytm jeszcze raz przechodzi po wszystkich kandydatach, a więc $N$ jest kolejnym czynnikiem. \\ \\
W kontekście tego algorytmu aproksymacyjnego nie rozważano formalnego dowodu poprawności. Dostarczono empiryczną gwarancję jakości - algorytm testowano na dużych zbiorach danych, wykazując, że algorytm konsekwentnie zwraca wyniki bliskie optimum w praktyce. 
% ALGORYTM APROKSYMACYJNY DO PROBLEMU IZOMORFIZMU PODGRAFU

\pagebreak
% ---------------- MINIMALNE ROZSZERZENIE MULTIGRAFU ---------------- %

% \section{Testy}
% \section{Podsumowanie}
\section{Bibliografia}

\bibliographystyle{plain}
\bibliography{refs}

\end{document}
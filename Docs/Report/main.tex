\documentclass[12pt]{article}
\usepackage[pdftex]{graphicx} % Required for inserting images
\usepackage[T1]{fontenc}
\usepackage[polish]{babel}
\usepackage{amsmath}
\usepackage{geometry}
\usepackage{float}
\usepackage{hyperref}
\usepackage{polski}
\usepackage[utf8]{inputenc}
\usepackage{gensymb}
\usepackage{geometry}
\usepackage{longtable}
\usepackage{array} % required for text wrapping in tables
\usepackage{algpseudocode}
\usepackage{algorithm}
\usepackage{indentfirst}
\usepackage{verbatim}
\usepackage{amsthm}
\usepackage{thmtools}
\theoremstyle{definition}
\newtheorem{definition}{Definicja}
\begin{document}

\title{
  \textbf{Teoria algorytmów i obliczeń} \\
  \large Projekt zaliczeniowy}

\author
{
    Piotr Jacak \\
    Jakub Kindracki \\
    Wiktor Kobielski \\
    Ernest Mołczan \\
    \\
    Koordynator: prof. dr hab. inż. Władysław Homenda
    \\
    \\
}

\date{Semestr zimowy 2025/2026}



\begin{figure}
    \centering
    \includegraphics[width=1\linewidth]{mini.png}
\end{figure}

\maketitle

\pagebreak
\tableofcontents
\pagebreak

% ---------------- WSTĘP ---------------- %
\section{Wstęp}
\label{sec:wstep} % Etykieta do późniejszych odwołań, np. \ref{sec:wstep}

Niniejsza praca stanowi sprawozdanie z projektu zrealizowanego w ramach przedmiotu \textbf{Teoria algorytmów i obliczeń}. Przedmiotem badań są algorytmy operujące na multigrafach, ze szczególnym uwzględnieniem problematyki izomorfizmu podgrafów oraz minimalnych rozszerzeń grafów.

Głównym celem projektu jest opracowanie, analiza teoretyczna oraz implementacja algorytmów rozwiązujących dwa ściśle powiązane problemy. Pierwszym z nich jest weryfikacja, czy dany multigraf $H$ jest izomorficzny z podgrafem multigrafu $G$. Drugim, kluczowym zagadnieniem, jest wyznaczenie \emph{minimalnego rozszerzenia} multigrafu $G$ do postaci $G'$, która zawiera co najmniej jeden podgraf izomorficzny z $H$.

Realizacja powyższych celów wymagała formalnego zdefiniowania oraz uzasadnienia kilku fundamentalnych pojęć. W pracy zaproponowano autorskie lub bazujące na literaturze definicje:
\begin{itemize}
    \item \emph{rozmiaru multigrafu},
    \item \emph{metryki} w zbiorze multigrafów,
    \item \emph{minimalnego rozszerzenia} multigrafu.
\end{itemize}
Pojęcia te stanowią podstawę do dalszej analizy algorytmicznej oraz oceny kosztu operacji.

W ramach pracy przeprowadzono analizę złożoności obliczeniowej opracowanych algorytmów. Zgodnie z założeniami projektu, w przypadku gdy algorytmy dokładne charakteryzują się złożonością wykładniczą, przedstawiono również propozycje algorytmów aproksymacyjnych o złożoności wielomianowej.

Niniejszy raport, oprócz formalnych definicji i analizy algorytmów, zawiera także opis przeprowadzonych testów obliczeniowych, dokumentację techniczną implementacji oraz wnioski końcowe.

\pagebreak
% ---------------- WSTĘP ---------------- %

% ---------------- DEFINICJE ---------------- %
\section{Definicje pojęć}
\label{sec:definicje}

% Definicje grafu, multigrafu, izomorfizmu, podgrafu izomorficznego, macierzy sąsiedztwa

\begin{definition}[Graf]
    Grafem nazywamy parę $G = (V, E)$, gdzie $V$ jest zbiorem wierzchołków, a $E \subseteq V \times V = \{(u, v) : u, v \in V \land u \neq v \}$ jest zbiorem krawędzi. Dla każdej pary wierzchołków $u, v \in V$ istnieje co najwyżej jedna krawędź łącząca wierzchołki $u$ i $v$.
\end{definition}

\begin{definition}[Multigraf]
    Multigrafem nazywamy graf, w którym pomiędzy dowolnymi dwoma różnymi wierzchołkami $u, v \in V$ może istnieć więcej niż jedna krawędź.
\end{definition}

\begin{definition}[Izomorfizm grafów]
    Dwa grafy $G_1 = (V_1, E_1)$ i $G_2 = (V_2, E_2)$ są izomorficzne, wtedy i tylko wtedy, gdy istnieje bijekcja $f: V_1 \to V_2$, taka że dla każdej krawędzi $(u, v) \in E_1$ zachodzi $(f(u), f(v)) \in E_2$. Definicja ta jest analogiczna dla multigrafów.
\end{definition}

\begin{definition}[Podgraf]
    Graf $H = (V_H, E_H)$ nazywamy podgrafem grafu $G = (V_G, E_G)$, wtedy i tylko wtedy, gdy $V_H \subseteq V_G$ oraz $E_H \subseteq E_G$. Definicja ta jest analogiczna dla multigrafów.
\end{definition}

\begin{definition}[Macierz sąsiedztwa]
    Macierzą sąsiedztwa multigrafu $G = (V, E)$ nazywamy macierz $A$, której pole $A_{uv} = k$, wtedy i tylko wtedy, gdy istnieje $k$ krawędzi $(u, v) \in E$. W przypadku gdy nie istnieje żadna krawędź pomiędzy wierzchołkami $u$ i $v$, to $A_{uv} = 0$. Dla zwykłych grafów, macierz sąsiedztwa jest macierzą binarną.
\end{definition}

\pagebreak
% ---------------- DEFINICJE ---------------- %

\section{Rozmiar multigrafu}
\section{Odległość multigrafów}
\section{Minimalne rozszerzenie multigrafu}
\section{Testy}
\section{Podsumowanie}
\section{Bibliografia}

\bibliographystyle{plain}
\bibliography{refs}

\end{document}